\chapter{Conclusions and Outlook} \label{sec:conclusions_and_outlook}

During the first year of this task, 
we surveyed the structure of the calibration workflows employed in the experiment, 
identified workflows suitable for integration into R$^3$, 
and explored modifications to the CMS data acquisition system for both the demonstrator setup and the final implementation in \Phasetwo.
For our next year, we plan on having a continued and more detailed follow-up survey (focusing on optimal statistics, inter-dependencies, validation etc.) of all calibrations (including offline). Such a survey will also focus more on the context of \Phasetwo calibrations and including the new subdetectors that will be installed for that upgrade: 
the MIP Timing Detector,
the High-Granularity Calorimeter,
and the new Silicon Tracker system.
We will also start evaluating the impact of the improved calibrations in the quality of online reconstruction.
Finally, we will focus on the final design and construction of the demonstrator prototype system, aiming for deployment and evaluation by the end of \Runthree. A dedicated performance versus price evolution study of the buffering medium (SSDs) is foreseen.