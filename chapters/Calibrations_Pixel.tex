\subsection{Pixel Tracker Calibrations}\label{sec:PixelCalib}

The CMS Silicon Pixel detector is integral to the High-Level Trigger (HLT) decision-making process, as it provides information for the seeding of the charged particle tracking algorithm, as well as crucial information for the determination of the primary and secondary interaction vertices parameters \cite{Dominguez:1481838,Adam_2021}.\\
This part of the report delves into the calibration prospects of the SiPixel detector, emphasising unused records, Front End Drive (FED) cabling, gain and quality calibrations, Cluster Position Estimator (CPE) conditions, and alignment practices.\\
Future prospects and recommendations for improving calibration workflows under the NGT framework are also discussed.\newline \newline 
Out of the 326 conditions in the HLT GT, 13 are related to the Silicon Pixel Tracker, of which 12 are relevant to Run 3 data-taking. From these, the following 8 are considered core and critical for data-taking:

\begin{table}[h!]
    \centering
    \begin{adjustbox}{max width=\textwidth}
    \begin{tabular}{p{2.5cm}|p{5.7cm}|p{1.5cm}|p{2.5cm}|p{3.0cm}}
        \textbf{Name} & \textbf{Record} & \textbf{Workflow} & \textbf{Frequency of Updates} & \textbf{Description} \\ \hline
      Lorentz Angle   & \texttt{SiPixelLorentzAngleRcd}                 & Manual & Typically a few times per year (mandatory when pixel HV changes) & Hall mobility per unit magnetic field strength\\ \hline
      Generic Errors  & \texttt{SiPixelGenErrorDBObjectRcd}             & Manual & Typically a few times per year (mandatory when pixel HV changes) & Uncertainty on the hit position \\ \hline
      1D-templates     & \texttt{SiPixelTemplateDBObjectRcd}             & Manual & Typically a few times per year (mandatory when pixel HV changes) & Cluster shape for hit position determination (1D)\\ \hline
      2D-templates    & \parbox[t]{5cm}{\texttt{SiPixel2DTemplateDBObjectRcd,}\\\texttt{numerator}} & Manual & Typically a few times per year (mandatory when pixel HV changes) & Cluster shape for hit position determination (2D)\\ \hline
      Bad channels    & \texttt{SiPixelQualityFromDbRcd}                & Manual & Typically a few times per year & Inventory of detector and readout defects \\ \hline
      Per-pixel Gains & \texttt{SiPixelGainCalibrationForHLTRcd}        & Manual & Once or twice per year & ADC to electron conversion factors \\ \hline
      Lorentz Angle   & \texttt{SiPixelLorentzAngleRcd,forWidth}        & Manual & Rarely (last update done in 2019) & Hall mobility per unit magnetic field strength \\ \hline
      Cabling map     & \texttt{SiPixelFedCablingMapRcd}                & Manual & Fixed for a given detector geometry & Readout to detector ID map \\ \hline
      Dead channels   & \texttt{SiPixelDetVOffRcd}                      & Manual & Not used & Inventory of unpowered channels \\ \hline
      Per-column Gains& \texttt{SiPixelGainCalibrationOfflineRcd}       & Manual & Not used & ADC to electron conversion factor \\ \hline
      Lorentz Angle   & \parbox[t]{5cm}{\texttt{SiPixelLorentzAngleRcd,}\\\texttt{fromAlignment}} & Manual & Not used & Hall mobility per unit magnetic field \\ \hline
      Bad channles    & \parbox[t]{5cm}{\texttt{SiPixelQualityFromDbRcd,}\\\texttt{forRawToDigi}} & Manual & Not used & Detector level defects \\ \hline
        
    \end{tabular}
    \end{adjustbox}
    \caption{Fundamental Silicon Strip Tracker Calibrations, ordered in terms of frequency of updates.}
    \label{tab:PixelCalibrations_critical}
\end{table}

\subsubsection{Unused Records in SiPixel Calibration}
Several records present in the HLT Global Tag remain unused at HLT but hold relevance for other operational aspects, such as online Data Quality Monitoring (DQM):
\begin{itemize}
    \item \texttt{SiPixelDetVOffRcd}: Lists detector IDs with HV or LV off. Payload is ideal or empty.
    \item \texttt{SiPixelGainCalibrationOfflineRcd}: Stores gain and pedestal values for offline analysis.
    \item \texttt{SiPixelLorentzAngleRcd ("from alignment")}: Contains Lorentz Angle offsets, outdated since 2017.
    \item \texttt{SiPixelTemplateDBObjectRcd ("0T")}: Used during 0T templated tracking for PixelRecHits.
\end{itemize}

\subsubsection{FED Cabling and Calibration Maps}
The \texttt{SiPixelFedCablingMapRcd} records cabling maps connecting FEDs to readout channels. While not a direct calibration, this record plays a vital role in detector topology and reconstruction. This was last updated in 2017 with the transition to the Pixel Phase-1 detector.

\subsubsection{Bad Components}
The \texttt{SiPixelQualityRcd} holds payloads containing the module identification numbers (\texttt{DetId}) and readout chips (ROC)s numbers associated with those \texttt{DetID}s that are considered "dead." It is used by the track reconstruction algorithm to identify regions in the detector where holes in the hit pattern along the trajectory are expected.\\

\subsubsection{Pixel Hit Position Reconstruction}

The position of a pixel cluster in the transverse (\(u\)) and longitudinal (\(v\)) directions on the sensor is determined by projecting the cluster onto the respective axis and summing the charge collected in pixels with the same coordinate \cite{Tracking_2014}. For single-pixel clusters, the position is the center of the pixel, corrected for Lorentz drift. For larger clusters, the hit position (\(u_{\text{hit}}\)) is computed using the relative charge in the first and last pixels of the projected cluster:

\begin{equation}
u_{\text{hit}} = u_{\text{geom}} + \frac{Q_{\text{last}} - Q_{\text{first}}}{2(Q_{\text{last}} + Q_{\text{first}})} |W_u - W_u^{\text{inner}}| - \frac{L_u}{2},
\end{equation}

where:
\begin{itemize}
    \item \(Q_{\text{first}}\), \(Q_{\text{last}}\): Charges in the first and last pixels of the cluster.
    \item \(u_{\text{geom}}\): Geometric center of the projected cluster.
    \item \(L_u/2 = D \tan \Theta_u^{\text{L}}/2\): Lorentz shift, with \(\Theta_u^{\text{L}}\) as the Lorentz angle and \(D\) as the sensor thickness.
    \item \(W_u^{\text{inner}}\): Geometrical width excluding the first and last pixels, zero for clusters smaller than three pixels.
    \item \(W_u = D \left| \tan (\alpha_u - \pi/2) + \tan \Theta_u^{\text{L}} \right|\): Charge width, derived from the track impact angle (\(\alpha_u\)) and Lorentz angle.
\end{itemize}

If no track is available, \(\alpha_u\) is assumed based on the particle's origin at the CMS detector center. The formula accounts for the partial charge deposition in the first and last pixels, estimating the extension of charge into these pixels (\(W_u - W_u^{\text{inner}}\)), which ranges between zero and twice the pixel pitch. This extension, combined with the relative charge, provides the edges of the charge distribution. The mean value of these edges, corrected for Lorentz drift, yields the cluster position.

For pixel barrel detectors, the Lorentz shift is approximately 59~\(\mu\)m.

\subsubsection{Pixel Lorentz Angle}
This condition is stored under the record  \texttt{SiPixelLorentzAngleRcd}.\\
Several variants are available: the unlabeled payload contains the value of Lorentz Angle (LA) per unit magnetic field in Tesla for the Barrel Pixel and Forward Pixel separately (due to different high voltage (HV) constants, geometry, etc.).\\
The two other labels present in the Global Tag are:
\begin{itemize}
\item \texttt{fromAlignment}: the LA offset generated by alignment (alternative to LA correction from templates).
\item \texttt{forWidth}: the LA for charge width estimate in the generic CPE (alternative to the same LA as for offset). 
\end{itemize}

\subsubsection{Pixel Templates}
The record \texttt{SiPixelTemplateDBObjectRcd}, contains templated information about cluster shapes used in the final fit step of the offline tracking algorithm for Pixel RecHits.\\
This record is currently not used at HLT as the generic variant of the cluster position estimation is used, but it is present in the online Global Tag for usage from other workflows.\\
Radiation exposure in the pixel detector significantly impacts charge collection and degrades the performance of the standard hit reconstruction algorithm, particularly in highly irradiated sensors. To address this, a template-based reconstruction algorithm compares the observed cluster charge distribution to precomputed templates generated using the \texttt{PIXELAV} simulation.\\
Templates are created by simulating particle tracks traversing pixel modules at various angles. Each pixel is subdivided into nine bins along the \(u\) (or \(v\)) axis, and the charge profile of the cluster is projected into an array of 13 pixels (or 23 for \(v\)). Clusters with excessively high charges, distorted by energetic delta rays, are excluded. The mean charge \(\bar{S}_{i,j}\) and RMS charge distributions for each bin are computed, along with other relevant cluster properties.\\
The observed charge \(P_i\) in each pixel \(i\) is compared to the expected template charge \(S_{i,j}\) to determine the likely bin \(j\) where the particle crossed the sensor. This is achieved by minimizing the \(\chi^2\) function:
\[
\chi^2(j) = \sum_i \left( \frac{P_i - N_j S_{i,j}}{\Delta P_i} \right)^2,
\]
where \(\Delta P_i\) is the RMS charge uncertainty, and \(N_j\) normalizes the cluster charge to the template:
\[
N_j = \frac{\sum_i \frac{P_i S_{i,j}}{(\Delta P_i)^2}}{\sum_i \frac{S_{i,j}^2}{(\Delta P_i)^2}}.
\]
This approach provides the best estimate of the hit position, even for complex multi-pixel clusters.\\

For single-pixel clusters, the hit position is corrected for Lorentz drift and radiation damage using the average residual bias. For multi-pixel clusters, the hit position is refined by approximating the template charge near the best \(j\)-bin as \((1-r)S_{i,j-1} + rS_{i,j+1}\) and minimizing \(\chi^2\) with respect to \(r\).\\
The reconstruction algorithm is validated on \texttt{PIXELAV} simulation \cite{Swartz:687440} samples used to generate the templates. By comparing reconstructed and true hit positions, residual biases are determined and accounted for in collision data. The RMS of these differences defines the uncertainty in the reconstructed hit position, ensuring robust performance across varying conditions.

\subsubsection{Generic Errors}
The record \texttt{SiPixelGenErrorDBObjectRcd} contains information about the uncertainties associated with cluster position estimation, used as input by the generic CPE algorithm. This record also includes a dedicated correction, known as the irradiation bias correction (IBC), which mitigates position estimation biases caused by radiation damage and detector ageing.

\subsubsection{Pixel Alignment}
Track-based alignment is a fundamental technique used to optimize the geometric description of the CMS pixel detector. By analyzing the trajectories of charged particles traversing the detector, this method minimizes the residual differences between measured hit positions and the predicted positions based on reconstructed tracks. This process accounts for misalignments arising from construction tolerances, operational stresses, and environmental effects, such as temperature variations and radiation damage. Accurate alignment is essential for ensuring precise track reconstruction, which is critical for physics analyses, as it directly impacts measurements of particle momentum and vertex position.
Several records are used to encode the geometry information in the HLT global Tag: 
\begin{itemize}
\item \texttt{TrackerAlignmentErrorRcd} (for alignment), 
\item \texttt{TrackerSurfaceDeformationRcd} (for surface deformation)  \item \texttt{GlobalPositonRcd} (defining the relative position of the subdetectors) 
\end{itemize}

Tracker alignment is tightly coupled with CPE conditions to mitigate Lorentz Angle miscalibrations \cite{CMS:2022ali}. The high-granularity offline PCL helps to optimize local biases arising from pixel local reconstruction miscalibrations but is prone to generate systematic biases affecting the track kinematics knows as weak modes \footnote{A class of systematic biases arising from the internal symmetries of the alignment problem, such as the cylindrical symmetry of the detector, or the fact that most tracks originate from a single region of space. This results in non-physical geometrical transformations, that leave though the track $\chi^{2}$ either unchanged or slightly changed, thus difficult to tackle by standard minimization algorithms}.
HLT reconstruction uses the Pixel CPE Fast algorithm (a variant of the generic algorithm compliant with heterogeneous computing), while offline operations rely on Pixel Template reconstruction. Discrepancies between these algorithms necessitate customized alignment procedures.

\begin{figure}[htbp]
   \centering
	\includegraphics[width=0.5\textwidth]{figures/pixel_alignment_sketch.png}
   \caption{Sketch showing the transverse view of the Phase-0 barrel pixel subdetector, made of successive layers of silicon modules. The alternating orientation of the modules within each layer is indicated by the triangles. The blue (grey) circles represent the reconstructed hit positions using incorrect (correct) Lorentz angles in the presence of a magnetic field . The grey curve corresponds to a track built from the hits that were reconstructed with the correct Lorentz angles. Hits reconstructed with incorrect Lorentz angles are displaced in a direction defined by the orientation of the module, increasing the residual distance between the hits and the track. \cite{CMS:2022ali}}
   \label{fig:pixelAlignment}
\end{figure}

\subsubsection{NGT Demonstrator Candidate Evaluation}

To summarise the main NGT calibration workflow candidates:
\begin{itemize}
    \item \textbf{Bad Components Masking}:
    \begin{itemize}
        \item Already implemented in PCL workflows and requires minimal statistics for updates.
        \item Demonstrated to improve track building in inside-out muon reconstruction.
    \end{itemize}
    \item \textbf{Pixel Alignment}:
    \begin{itemize}
        \item Enhancing alignment directly impacts B-tagging and physics trigger performance.
        \item Requires tailored workflows to address weak modes effectively.
    \end{itemize}
\end{itemize}

The recommendations in the context of the NGT calibration workflow are:
\begin{itemize}
    \item Prioritize the inclusion of bad components masking in the NGT demonstrator workflow due to its operational simplicity and demonstrated impact.
    \item Develop alignment calibration workflows that directly integrate HLT tracks to ensure compatibility.
    \item Evaluate the feasibility of frequent updates for gains and CPE conditions to maintain precision.
\end{itemize}