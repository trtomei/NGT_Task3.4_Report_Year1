\chapter{Report's goals}

\section{Introduction}

\subsection{Calibrations General}
%%% Some introductory text to explain the main concepts.

\subsection{CMS Calibration Workflows}
% TODO: General introduction

% TODO: explain general AlCa: calibration vs condition, records, tags, etc.

% Latest GTs:
326 conditions (records/tags) in the latest HLT Global Tag (\texttt{140X\_dataRun3\_HLT\_v3}) used during LHC Run 3 p-p data taking in 2024.

% NOTE: Introduce prompt/offline in a similar manner?

\subsubsection{PCL}

\begin{figure}[h!]	
\centering
\includegraphics[width=\textwidth]{figures/PCL.jpg} %\hfill
\caption{Prompt Calibration Loop (PCL)}
\label{fig:PCL}
\end{figure}

\subsubsection{O2O}
\subsubsection{Other}

%%% Here we should add something about the automation framework






















































% The objective of \ngt{}'s \task{3.1.1} is to reconstruct all events received
% from the L1T at a rate of \SI{750}{\kilo\hertz} during Run-\num{5} data taking,
% eventually with an \emph{Offline-like} quality reconstruction. The
% reconstruction should enable some ``quick and dirty'' analyses using the
% reconstructed objects, without requiring additional steps. In order to
% understand where we are in terms of computing and physics performance, we
% decided to conduct a series of measurements using different scenarios, as
% explined in \ref{sec:measurements}. Considering a
% fixed budget for the farm in Run-\num{4} and Run-\num{5}, we need to determine
% the speed-up factor needed to be able to process the whole L1TA input rate with
% the required quality.
% 
% \section{Measurements}
% \label{sec:measurements}
% 
% As part of the first-year milestones, \task{3.1.1} is responsible for
% delivering a report on the performance of online reconstruction. This report
% will address identified bottlenecks, propose targeted improvements, and outline
% the necessary features for the generic CMS Structure of Arrays (\soa{}). To
% achieve this, we have begun conducting measurements using recent \CMSSW*{}
% releases. Our focus is on three specific performance metrics:
% 
% \begin{itemize}
% \item
%   \textbf{Measure the current performance of the \emph{simplified HLT Phase2
%     menu} for HL-LHC}: This will evaluate the present performance in terms of
%     computing speed and memory usage, with an aim to extrapolate these metrics
%     to 2030, the expected start of HL-LHC operations, and further down to 2032,
%     the beginning of Run-\num{5} operations. This will be done using both a
%     sample of \emph{TTbar} simulated events from the \textbf{Spring24} ongoing
%     campaign under 200PU conditions (and 140PU, if available), and a
%     specifically created \textit{skim} of L1 accepted events over an input
%     sample of \textit{MinimumBias} events, at 200PU.
% \item
%   \textbf{Assess the current performance of the \emph{offline Phase2
%     reconstruction}}: This will be done using both a sample of \emph{TTbar}
%     simulated events from the \textbf{Spring24} ongoing campaign under 200PU
%     conditions (and 140PU, if available), and a specifically created
%     \textit{skim} of L1 accepted events over an input sample of
%     \textit{MinimumBias} events, at 200PU. We will then project this
%     performance to 203{0,2}, considering anticipated improvements in CHF/HS06
%     and the likely shift of some algorithms and reconstruction processes to
%     accelerators. This analysis will help identify the necessary speedup factor
%     required to handle the full L1A rate directly during HLT processing using
%     an \textbf{offline-like reconstruction}, which is the final goal of this
%     task.
% \item
%   \textbf{Account for ongoing development in the \emph{offline Phase2
%   reconstruction}}: Since it is still evolving and lacks some
%   improvements present in the \emph{Run-3 reconstruction}, we also plan
%   to measure the performance of the \emph{Run-3 reconstruction} on a
%     simulated \textit{TTbar} sample with roughly 60PU. We will then identify which
%   modules from the \emph{Run-3 scenario} are missing in the
%   \emph{offline Phase2} sequence, estimate their
%   significance---particularly regarding CPU usage---extrapolate this
%   data to 200PU conditions, and incorporate it into our previous
%   performance measurements.
% \end{itemize}
% 
% All measurements have been done using \CMSSW[14][2][0][1]{}. A few additional patches have been necessary in order to correctly re-process the events. The patches fixed genuine bugs in the Phase\num{2} HLT Menu that were discovered while performing these measurements. Those patches have been submitted upstream to the main \CMSSW*{} release as pull requests (\href{https://github.com/cms-sw/cmssw/pull/46019}{46019} and \href{https://github.com/cms-sw/cmssw/pull/46054}{46054}) and integrated.
% 
% All samples used have been specifically produced to optimize the performance of the Phase\num{2} HLT reconstruction and of the offline reconstruction. Detailed instructions on how to create the samples and about their specificities are avaialble at \href{https://cms-ngt-hlt.docs.cern.ch/Task311/2024/Recipes/}{this link}\cite{ngt-recipes}.
% 
% Once all measurements are completed and the data is available, we will
% be in a better position to propose the next steps for development.
% Specifically, we will understand the extent of \textit{disruption} required to
% meet the highly ambitious goals outlined in this task.
% 
% \subsection{Pie resources}
% \label{subsec:pie-resources}
% 
% All pie-chart resources are usually available at
% \href{https://rovere.web.cern.ch/rovere/Phase2_HLT/circles/web/piechart.php}{this
% link}\cite{ngt-pies}. The measurements that are related to the \ngt{} project are
% usually collected under the \texttt{NGT\_\-Mea\-sure\-ments} folder.
% 
% \begin{nbbox}
% Some of the measurements collected at the link above are the results of several
% trials and errors. The measurements that are meant to be representative of each
% specific condition will be linked in the proper sections.
% \end{nbbox}
