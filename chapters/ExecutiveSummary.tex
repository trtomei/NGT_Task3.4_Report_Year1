%!TEX root = ../main.tex
\chapter{Executive Summary}

The Real-time Reconstruction Revolution (\Rthree) component of the Next-Generation Triggers project envisions real-time, high-quality event reconstruction and recording of the full Level-1 Trigger output of the CMS experiment.
Task 3.4 of the \Rthree effort, \emph{Optimal Calibrations}, focuses on
designing an optimised workflow with improved calibration, automated processes, and data buffering to make that vision a reality,
with a demonstrator prototype slated for deployment in the last year of \Runthree.
In the first year of this task we 
reviewed the structure of calibration workflows used in the experiment, 
identified candidate workflows suitable for inclusion in \Rthree, and
discussed the adaptation of the data acquisition (DAQ) system both for the demonstrator system and
for the final system in \Phasetwo.

To identify candidates suitable for inclusion in \Rthree, 
we studied the full set of experimental conditions used on online operations and 
requested the detector experts to fill a survey detailing the conditions deemed critical for data-taking.
Those efforts led us to identify the following initial set of candidates:
reconstruction of the beam-beam interaction luminous region;
silicon pixel tracker alignment and
silicon strip tracker bad component masking;
electromagnetic calorimeter crystal transparency measurement, 
and
hadronic endcap and forward calorimeter gains and electronic pulse pedestals.
These choice of those calibrations is due to their frequent updates, 
operational feasibility, 
and significant contributions to improving real-time event reconstruction.

We also studied the embedding of the Optimal Calibration system into the CMS DAQ architecture. 
During \Runthree, the DAQ system features a pipeline structure with different computer nodes dedicated to subdetector readout,
event building and
event filtering. 
The proposed integration involves using local buffering to temporarily store events, enabling calibration updates within 8--12 hours before final processing.
Preliminary calculations point that the system can scale for \Phasetwo with considerations for storage capacity, node allocation, and data transfer to permanent storage. 
A minimal demonstrator is envisioned for deployment, focusing on the prioritised calibration tasks to demonstrate feasibility of the system.